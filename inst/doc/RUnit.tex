% -*- mode: noweb; noweb-default-code-mode: R-mode; -*-
%
%  $Id: RUnit.Rnw,v 1.20 2009/03/16 16:21:04 burgerm Exp $
%
%
%\VignetteIndexEntry{RUnit primer}
%\VignetteKeywords{Unit Testing, Code Inspection, Programming}
%\VignetteDepends{methods, splines}
%\VignettePackage{RUnit}
\documentclass[12pt, a4paper]{article}

%\usepackage{amsmath,pstricks}
\usepackage{hyperref}
%\usepackage[authoryear,round]{natbib}


%\parskip=.3cm
\oddsidemargin=.1in
\evensidemargin=.1in
\headheight=-.3in

\newcommand{\scscst}{\scriptscriptstyle}
\newcommand{\scst}{\scriptstyle}
\newcommand{\Rfunction}[1]{{\texttt{#1}}}
\newcommand{\Robject}[1]{{\texttt{#1}}}
\newcommand{\Rpackage}[1]{{\textit{#1}}}

%\makeindex
%
\usepackage{/mnt/local/R/hardy/R-2.9.0/lib/R/share/texmf/Sweave}
\begin{document}
\title{RUnit - A Unit Test Framework for R}
\author{Thomas K\"onig, Klaus J\"unemann, and Matthias Burger\\Epigenomics AG}
\maketitle
\tableofcontents
\section*{Abstract}
\label{section:abstract}
Software development for production systems presents a challenge to the development team as the quality of the coded package(s) has to be constantly monitored and verified. We present a generic approach to software testing for the R language modeled after successful examples such as JUnit, CppUnit, and PerlUnit. The aim of our approach is to facilitate development of reliable software packages and provide a set of tools to analyse and report the software quality status. The presented framework is completely implemented within R and does not rely on external tools or other language systems. The basic principle is that every function or method is accompanied with a test case that queries many calling situations including incorrect invocations. A test case can be executed instantly without reinstalling the whole package - a feature that is necessary for parallel development of functionality and test cases. On a second level one or more packages can be tested in a single test run, the result of which is reported in an well structured test protocol.
To verify the coverage of the test framework a code inspector is provided that monitors the code coverage of executed test cases. The result of individual test invocations as well as package wide evaluations can be compiled into a summary report exported to HTML. This report details the executed tests, their failure or success, as well as the code coverage. Taking it one step further and combining the build system with a development and release procedure with defined code status description this approach opens the way for a principled software quality monitoring and risk assessment of the developed application.
For our code development we have utilised the described system with great benefit w.r.t.\ code reliability and maintenance efforts in a medium sized development team.

\section{Introduction}
The importance of software testing can hardly be overrated. This
is all the more true for interpreted languages where not even a compiler
checks the basic consistency of a program. Nonetheless, testing is
often perceived more as a burden than a help by the
programmer. Therefore it is necessary to provide tools that make the
task of testing as simple and systematic as possible. The key goal of
such a testing framework should be to promote the creation and
execution of test cases to become an integral part of the software
development process. Experience shows that such a permanently repeated
code - test - simplify cycle leads to faster and more successful
software development than the usually futile attempt to add test cases
once the software is largely finished. This line of thought has been
pushed furthest by the Extreme Programming
\cite{xp} and Test-First paradigms
where test cases are viewed as the essential guidelines for the
development process. These considerations lead to various requirements
that a useful testing framework should satisfy:
\begin {itemize}
\item {Tests should be easy to execute.}
\item {The results should be accessible through a well structured test
    protocol.}
\item{It should be possible to execute only small portions of the test
    cases during the development process.}
\item{It should be possible to estimate the amount of code that is
    covered by some test case.}
\end {itemize}


%\paragraph{Background}
%\label{paragraph:Background}
Testing frameworks that address these aspects have been written in a
variety of languages such as Smalltalk, Java, C++ and Python. In
particular, the approach described in \cite{beck} has turned out to be
very successful, leading -- among others -- to the popular JUnit
library for Java \cite{junit}, which has
been ported to many other languages (see \cite{xp} for an extensive list of testing
frameworks for all kinds of languages). Accordingly, the RUnit package (available at sourceforge \cite{runit-sf}) is our
version of  porting JUnit to R, supplemented by additional
functionality to inspect the test coverage of some function under question.

%\paragraph{Motivation}
%\label{paragraph:Motivation}
One may wonder why R would need yet another testing framework
even though the standard method, namely executing {\it R CMD check} on ones complete package at the shell prompt, is widely accepted and applied.
We think, however, that the RUnit approach is more in line with the above listed
requirements and can be seen as a complement to the existing process in that:
\begin{itemize}
\item{test cases are called and executed from the R prompt}
\item{the programmer decides which result or functionality to put under testing, e.g.\
      formating issues of textual output do not need to matter}
\item{test and reference data files need not be maintained separately but are combined into one file}
\item{test cases need not be limited to testing/using functionality from one package checked at a time}
\end{itemize}
Moreover, testing frameworks based on JUnit ports seem
to have become a quasi standard in many programming languages. Therefore,
programmers new to R but familiar with other languages might
appreciate a familiar testing environment. And finally, offering more
than one alternative in the important field of code testing is certainly not a bad idea and could turn out useful.

Before explaining the components of the RUnit package in detail,
we would like to list some of the lessons learned in the attempt of
writing useful test suites for our software (a more complete collection
of tips relating to a Test-First development approach can be found in \cite{tfg}):
\begin{itemize}

\item {Develop test cases parallel to implementing your
    functionality. Keep testing all the time (code - test - simplify
    cycle). Do not wait until the software is complete and attempt to
    add test cases at the very end. This typically leads to crappy
    and incomplete test cases.}

\item{Distinguish between unit and integration tests: Unit tests
    should be as small as possible and check one unit of functionality
    that cannot be further decomposed. Integration tests, on the other
    hand, run through a whole analysis workflow and check the
    interplay of various software components.}

\item{Good test coverage enables refactoring, by which a
        reorganisation of the implementation is meant. Without testing the
        attitude {\it `I better do not touch this code anymore`} once some piece
        of software appears to be working is frequently
        encountered. It is very pleasing and time-saving just to run a
        test suite after some improvement or simplification of the
        implementation to see that all test cases are still passing
        (or possibly reveal some newly introduced bug). This
        refactoring ability is a key benefit of unit testing leading
        not only to better software quality but also to better design.}

\item{Do not test internal functions but just the public interface of
    a library. Since R does not provide very much language support for this
    distinction, the first step here is to clarify which
    functions are meant to be called by a user of a package and which are
    not (namespaces in R provide a useful directive for making this distinction, if
    the export list is selected carefully and maintained). 
    If internal functions are directly tested, the ability of
    refactoring gets lost because this typically involves
    reorganisation of the internal part of a library.}

\item {Once a bug has been found, add a corresponding test case.}

\item{We greatly benefitted from an automated test system: A
    shell script, running nightly, checks out and installs all relevant packages.
    After that all test suites are run and the resulting test protocol is stored
    in a central location. This provides an excellent overview over the current
    status of the system and the collection of nightly test protocols documents
    the development progress.}

\end{itemize}


\section{The RUnit package}
\label{section:RUnitPackage}
This section contains a detailed explanation of the RUnit package and
examples how to use it. As has already been mentioned the package
contains two independent components: a framework for test case
execution and a tool that allows to inspect the flow of execution
inside a function in order to analyse which portions of code are
covered by some test case.
Both components are now discussed in turn.

\subsection{Test case execution}
\label{subsection:Testcaseexecution}
The basic idea of this component is to execute a set of test functions
defined through naming conventions, store whether or not the test
succeeded in a central logger object and finally write a test protocol
that allows to precisely identify the problems.

{\bf Note, that RUnit - by default - sets the version for normal, and all other RNGs to 'Kinderman-Ramage', and 'Marsaglia-Multicarry', respectively. If you like to change these defaults please see {\tt ?defineTestSuite} for argument 'rngNormalKind' and 'rngKind'.}

As an example consider a function that converts centigrade to
Fahrenheit:
\begin{Sinput}
c2f <- function(c) return(9/5 * c + 32)
\end{Sinput}
A corresponding test function could look like this:
\begin{Sinput}
test.c2f <- function() {
  checkEquals(c2f(0), 32)
  checkEquals(c2f(10), 50)
  checkException(c2f("xx"))
}
\end{Sinput}
The default naming convention for test functions in the RUnit package is {\tt test...} as is standard in JUnit. To perform the actual checks that the function to be tested works correctly a set of functions called {\tt check ...} is provided. The purpose of these {\tt check} functions is two-fold: they make sure that a possible failure is reported to the central test logger so that it will appear properly in the final test protocol and they are supposed to make explicit the actual checks in a test case as opposed to other code used to set up the test scenario. Note that {\tt checkException} fails if the passed expression does not generate an error. This kind of test is useful to make sure that a function correctly recognises error situations instead of silently creating inappropriate results. These check functions are direct equivalents to the various {\tt assert} functions of the JUnit framework. More information can be found in the online help.

Before running the test function it is necessary to create a test suite which is a collection of test functions and files relating to one topic. One could, for instance, create one test suite for one R package. A test suite is just a list containing a name, an array of absolute directories containing the locations of the test files, a regular expression identifying the test files and a regular expression identifying the test functions. In our example assume that the test function is located in a file {\tt runitc2f.r} located in a directory {\tt /foo/bar/}. To create the corresponding test suite we can use a helper function:
\begin{Sinput}
testsuite.c2f <- defineTestSuite("c2f",
                 dirs = file.path(path.package(package="RUnit"),
                            "examples"),
                 testFileRegexp = "^runit.+\\.r",
                 testFuncRegexp = "^test.+",
                 rngKind = "Marsaglia-Multicarry",
                 rngNormalKind = "Kinderman-Ramage")
\end{Sinput}
All that remains is to run the test suite and print the test protocol:
\begin{Sinput}
testResult <- runTestSuite(testsuite.c2f)
printTextProtocol(testResult)
\end{Sinput}
The resulting test protocol should be self explanatory and can also be printed as HTML version. See the online help for further information.
Note that for executing just one test file there is also a shortcut in order to make test case execution as easy as possible:
 \begin{Sinput}
runTestFile(file.path(path.package(package="RUnit"),
            "examples/runitc2f.r"))
\end{Sinput}

The creation and execution of test suites can be summarised by the following recipe:
\begin{enumerate}
\item{create as many test functions in as many test files as necessary }
\item{create one or more test suites using the helper function {\tt defineTestSuite}}
\item{run the test suites with {\tt runTestSuite}}
\item{print the test protocol either with {\tt printTextProtocol} or with {\tt printHTMLProtocol} (or with a generic method like {\tt print} or {\tt summary})}
\end{enumerate}

 We conclude this section with some further comments on various aspects of the test execution framework:
\begin{itemize}
\item{A test file can contain an arbitrary number of test functions. A test directory can contain an arbitrary number of test files, a test suite can contain an arbitrary number of test directories and the test runner can run an arbitrary number of test suites -- all resulting in one test protocol. The test function and file names of a test suite must, however,  obey a naming convention expressible through regular expressions.
    As default test functions start with {\tt test} and files with {\tt runit}.}
\item{RUnit makes a distinction between failures and error. A failure occurs if one of the check functions fail (e.g.~{\tt checkTrue(FALSE)} creates a failure). An error is reported if an ordinary R error (usually created by {\tt stop}) occurs.}
\item{Since version 0.4.0 there is a function {\tt DEACTIVATED} which
can be used to deactivate test cases temporarily. This might be useful
in the case of a major refactoring. In particular, the deactivated
test cases are reported in the test protocol so that they cannot fall
into oblivion.}
\item{The test runner tries hard to leave a clean R session behind. Therefore all objects created during test case execution will be deleted after a test file has been processed.}
\item{In order to prevent mysterious errors the random number generator is reset to a standard setting before sourcing a test file. If a particular setting is needed to generate reproducible results it is fine to configure the random number generator at the beginning of a test file. This setting applies during the execution of all test functions of that test file but is reset before the next test file is sourced.}
\item{In each source file one can define the parameterless functions {\tt .setUp()} and {\tt .tearDown()}.
which are then executed directly before and after each test function. This can, for instance, be used to control global settings or create addition log information.}

\end{itemize}



\subsection{R Code Inspection}
\label{subsection:RCodeInspection}
The Code Inspector is an additional tool for checking detailed test case coverage and getting profiling information.
It records how often a code line will be executed. We utilise this information for improving our test cases, because we can identify code lines not executed by the current test case code.
The Code Inspector is able to handle S4 methods.
During the development of the Code Inspector, we noticed, that the syntax of R is very flexible.
Because our coding philosophy has an emphasis of maintenance and a clear style, we developed style guides for our R coding.
Therefore, one goal for the Code Inspector was to handle our coding styles in a correct manner.
This leads to the consequence that not all R expression can be handled correctly.
In our implementation the Code Inspector has two main functional parts.
The first part is responsible for parsing and modifying the code of the test function.
The second part, called the Tracker, holds the result of the code tracking.
The result of the tracking process allows further analysis of the executed code.

\subsubsection{Usage}
The usage of the Code Inspector and the Tracker object is very simple. The following code snippet is an example:
\begin{Schunk}
\begin{Sinput}
> library(RUnit)
> foo <- function(x) {
+     x <- x * x
+     x <- 2 * x
+     return(x)
+ }
> test.foo <- function() {
+     checkTrue(is.numeric(foo(1:10)))
+     checkEquals(length(foo(1:10)), 10)
+     checkEqualsNumeric(foo(1), 2)
+ }
> bar <- function(x, y = NULL) {
+     if (is.null(y)) {
+         y <- x
+     }
+     if (all(y > 100)) {
+         y <- y - 100
+     }
+     res <- x^y
+     return(res)
+ }
> track <- tracker()
> track$init()
> a <- 1:10
> d <- seq(0, 1, 0.1)
> resFoo <- inspect(foo(a), track = track)
> resBar <- inspect(bar(d), track = track)
> resTrack <- track$getTrackInfo()
> printHTML.trackInfo(resTrack)
\end{Sinput}
\end{Schunk}

Note, that the tracking object is an global object and must have the name {\tt track}.
The {\tt inspect} function awaits a function call as argument and executes and tracks the function.
The results will be stored in the tracking object.
The result of the function (not of the Tracker) will be returned as usual.
The tracking results will received by tr\$getResult().
With {\tt printHTML} the result of the tracking process will be presented as HTML pages.

\subsubsection{Technical Details}
The general idea for the code tracking is to modify the source code of the function.
Therefore, we use the {\tt parse} and {\tt deparse} functions and the capability of R to generate functions on runtime.
To track the function we try to include a hook in every code line.
That hook calls a function of the tracked object.
The information of the tracking will be stored in the closure of the tracking object (actually a function).
Because the R parser allows very nested expressions, we didn't try to modify every R expression.
This is a task for the future.
A simple example for the modifying process is as follow:\\
original:
\begin{Schunk}
\begin{Sinput}
> foo <- function(x) {
+     y <- 0
+     for (i in 1:x) {
+         y <- y + x
+     }
+     return(y)
+ }
\end{Sinput}
\end{Schunk}
modified:
\begin{Schunk}
\begin{Sinput}
> foo.mod <- function(x) {
+     track$bp(1)
+     y <- 0
+     track$bp(2)
+     for (i in 1:x) {
+         track$bp(4)
+         y <- y + x
+     }
+     track$bp(6)
+     return(y)
+ }
\end{Sinput}
\end{Schunk}

Problematic code lines are:

\begin{Schunk}
\begin{Sinput}
> if (any(a == 1)) {
+     print("do TRUE")
+ } else print("do FALSE")
\end{Sinput}
\end{Schunk}
This must be modified to
\begin{Schunk}
\begin{Sinput}
> if (any(a == 1)) {
+     track$bp(2)
+     print("do TRUE")
+ } else {
+     track$bp(3)
+     print("do FALSE")
+ }
\end{Sinput}
\end{Schunk}
The problem is the \textit{else} branch, that cannot be modified in the current version.

\section{Future Development Ideas}
Here we briefly list -- in an unordered manner -- some of the avenues for future development we or someone interested in this package could take:
\begin{itemize}
\item{extend the {\tt checkEquals} function to handle complex S4 class objects correctly in comparisons. To this end R core has modified check.equal to handle S4 objects.}
\item{reimplement the internal structures storing the test suite as well as the test result data as S4 classes.}
\item{record all warnings generated during the execution of a test function.}
\item{add tools to create test cases automatically. This is a research project but -- given the importance of testing -- worth the effort. See \cite{junit} for various approaches in other languages.}
\item{improve the export of test suite execution data e.g.~by adding XML data export support.}
\item{add some evaluation methods to the code inspector e.g.~use software metrics to estimate standard measures of code quality, complexity, and performance.}
\item{overcome the problem of nested calls to registered functions for code inspection.}
\item{allow automatic registration of functions \& methods.}
\end{itemize}

\begin{thebibliography}{99}
% \bibliographystyle{plainnat}
\bibitem{xp} http://www.xprogramming.com
\bibitem{beck} http://www.xprogramming.com/testfram.htm
\bibitem{junit} http://www.junit.org/
\bibitem{tfg} http://www.xprogramming.com/xpmag/testFirstGuidelines.htm
\bibitem{runit-sf} https://sourceforge.net/projects/runit/
\end{thebibliography}

\end{document}






